\documentclass[a4paper,
			   11pt,
			   ngerman, 
			   ]{scrreprt}

\usepackage[left=2.5cm,right=2.5cm,top=2.5cm,bottom=2.5cm]{geometry}
\usepackage{lipsum}
\usepackage[utf8]{inputenc}
\usepackage{csquotes}
\usepackage{graphicx}
\usepackage[section]{placeins}
\usepackage{fontenc}
\usepackage[automark]{scrlayer-scrpage}
\usepackage[ngerman]{babel}
\usepackage[natbib=true,style=numeric, sorting=none]{biblatex}
\usepackage{svg}
\usepackage{setspace}
\setstretch{1.5}


%%%%%%%%% KOPFZEILE %%%%%%%%%%%%%%%%%%%%%
\pagestyle{scrheadings} 
\chead*{\headmark}
\cfoot*{\pagemark}

\addbibresource{ref/ref.bib}

\publishers{
	\begin{tabular}{rl} %jedes C für eine spalte c für centered
		Erstprüfer: & Prof. Dr. Christian Pfitzner \\
		Abgabedatum: & 20.06.2023 \\
	\end{tabular}
}


\begin{document}

	%%%%%%%%% KOPF %%%%%%%%%%%%%%%%%%%%%%%%%%
	\title{Seminararbeit}
	\subtitle{Sensorik-gestützte Robotik-Systeme zur Automatisierung landwirtschaftlicher Prozesse}
	\author{Nico Elsner\\
	Studiengang: Robotik\\
	Technische Hochschule Ingolstadt}
	\date{\today}
	\maketitle
	\tableofcontents
	\thispagestyle{empty}
	%%%%%%%%%% EINLEITUNG %%%%%%%%%%%%%%%%%%%
	\newpage
	\setcounter{page}{1}
	\chapter{Einleitung}
	\section{Problemstellung und Motivation}
		In den vergangenen Jahrzehnten entwickelte sich durch das steigende Bevölkerungswachstum ein weltweiter Mangel an Ressourcen, vor allem an Nahrungsmitteln. Die Folge hiervon sind Massentierhaltung, Monokulturen und Gewächshausplantagen. Der Trend geht zur Perfektion der Erträge, bei möglichst geringem Aufwand.\\
	In anderen Bereichen, wie der Automobilindustrie ist die Frage nach schnellen, zuverlässigen Massenproduktionen bereits seit mehreren Jahrzehnten diskutiert. Vorallem die Automobilindustrie hat bereits früh begonnen sich mit Automatisierung zu beschäftigen. Erste Ansätze gab es hier bereits 1913 durch Henry Ford, welcher die ersten Autos auf dem Fließband zusammenbauen ließ. Dieses System wurde über die Jahrzehnte soweit ausgebaut, dass heutzutage bereits viele Industrieroboter die Bandarbeit nahezu komplett autonom erledigen.\\
	Auch in der Landwirtschaft gibt es bereits Ansätze, um die Arbeit zu industrialisieren.
	\section{Einführung in das Thema der Arbeit}
	In der Landwirtschaft gibt es viele Prozesse, die immer gleich sind. Solche
Aufgaben sind perfekt dafür geeignet, um mithilfe von sensorbasierter Robotik
automatisiert zu werden. Zudem hat man bei Feldern, Äckern, etc. immer feste
Ortspunkte und kann somit sehr gut mit den globalen GPS Koordinaten arbeiten.
Im Folgenden wird auf die Grundlagen der benötigten Sensoren, sowie deren
Flächenabdeckungsmöglichkeiten eingegangen. Zudem werde ich einige bereits
existierende Robotersysteme vorstellen und einen Ausblick geben, wie diese auf
Industriegröße skaliert werden können und wo sie bereits Anwendung finden. Im
weiteren Verlauf wird dann noch der immer weitverbreitetere Aspekt der
künstlichen Intelligenz aufgegriffen und der Bezug zur Landwirtschaft
hergestellt. Das Ende der Arbeit fässt die gesammelten Erkenntnisse zusammen
und gibt eine Zukunftsprognose zum Thema Automatisierung der Landwirtschaft
durch sensorbasierte Robotik.
	
	%%%%%%%%%%% ÜBERGREIFENDE SYSTEME %%%%%%%
	\chapter{Systeme}
		\section{Drohnen}
		In der Landwirtschaft geht es häufig darum, Sachen (Flüssigkeiten,Samen,etc.) auf dem Feld auszubringen. Auf die herkömmliche Art macht das ein Bauer, mithilfe eines Traktors. Je nach Aufgabe und Feldgröße benötigt ein Landwirt hierfür mehrere kostbare Stunden. Die Ausbringung von Flüssigkeiten wäre ein typischer Anwendungsfall für eine Drohne.\\

\begin{figure}[ht]
	\centering
	\includegraphics[width=0.5\textwidth]{bilder/drohne_im_sprüheinsatz.png}
	\caption[Sprühdrohne]{Sprühdrohne}
	\label{fig:sprühdrohne}
\end{figure}

Die Drohne kann hierbei aus geringer Höhe die entsprechenden Flüssigkeiten über eine Düse ausbringen. Das kann durch Routenplanung komplett automatisiert passieren. Lediglich ein Flüssigkeitstank ist hier benötigt, welcher jedoch innerhalb weniger Minuten vom Landwirt an einer fest vorgegebenen Stelle platziert werden kann. Somit hat die Drohne einen festen Bezugspunkt für das Auffüllen der Flüssigkeit, ähnlich einer Ladestation eines Rasenmähers.

		\section{Schienensysteme}
		Unter Landwirtschaft zählen jedoch nicht nur klassische wie Äcker, Maisfelder, Getreidefelder, sondern unter anderem auch Gewächshäuser. Auch in Gewächshäusern gibt es viele Aufgaben, welche durch Sensoren ersetzt, bzw. sogar verbessert werden können.\\
Hier gibt es bereits ein bekanntes Beispiel aus der Heimbeet-Szene:\\

\begin{figure}[h]
	\centering
	\includegraphics[width=0.5\textwidth]{bilder/farmbot.png}
	\caption[Farmbot]{Farmbot}
	\label{fig:farmbot}
\end{figure}

Der Farmbot ist ein Schienensystem, was von der Mechanik stark an einen 3D-Drucker erinnert. Verkauft wird hierbei von dem Unternehmen nur die Hardware, sprich Schienen, Motoren, Adapter, Verkabelung und ein Raspberry Pi für die Software.\\
Die Software ist open-source, also frei im Internet für alle zugänglich, was mehrere Vorteile mit sich bringt:\\
Die Software wird von jedem User gedownloaded und eventuell umgeschrieben. Das bedeutet die Software wird kontinuierlich Probe gelesen. Außerdem können User die Software verändern und anpassen, etwaige Fehler beheben oder Performance-Verbesserungen vornehmen.
		\section{Roboterfahrzeuge}
		Ein weiteres weit verbreitetes Konzept sind Roboterfahrzeuge mit Reifen- oder
Rollenantrieb. Diese werden hauptsächlich bei niedrig wachsenden Sorten zur
Erkennung von Schädlingen oder beschädigten Pflanzen genutzt. Dies geschieht
durch hochauflösende Kamerasysteme und der Verarbeitung der Ergebnisse, zum
Teil mithilfe von künstlicher Intelligenz.

\begin{figure}[h]
	\centering
	\includegraphics[width=0.7\textwidth]{bilder/farmdroid_fd20.png}
	\caption[Farmdroid FD20]{Farmdroid FD20\cite{Wurst}}
	\label{fig:farmdroid_fd20}
\end{figure}

Der Farmdroid FD20(Abbildung \ref{fig:farmdroid_fd20}) arbeitet hierbei komplett autonom und klimaneutral durch
Solarpanels. Diese Solarpanels können bei guter Sonneneinstrahlung mehr als
genug Energie für den Roboter liefern. Durch zwei zusätzliche Lithium-Ionen
Akkus ist es möglich, dass der Farmdroid auch nach Sonnenuntergang autark
weiterarbeitet, was einen Dauerbetrieb ermöglicht und damit extrem effizient
ist.\cite{jungwirth2022arbeitszeitbedarf}\\ Die Aufgabe des modernsten
Agrarroboters der Welt\cite{donaukurier2022} ist die Saat und die
Unkrautvernichtung. Laut Hersteller hat sich der Farmdroid bereits nach bis zu
1,5 Jahren amortisiert und arbeitet ab diesem Zeitpunkt aufgrund der
Solarpanels mit extrem niedrigen Unterhaltskosten. Zudem ist der Roboter
aufgrund der zwei GNSS Empfänger und eines RTK Korrektursignals, welches in der
Basisstation erzeugt wird, sehr genau.\cite{jungwirth2022arbeitszeitbedarf}\cite{spykman2023wirtschaftlichkeitsbewertung}

	%%%%%%%%%%% BENÖTIGTE SENSOREN %%%%%%%%%%
	\chapter{Sensorik}
	
		\section{Temperatursensor}
		Wichtig für das Wachstum von Pflanzen ist natürlich die Temperatur. Je nach
Herkunft benötigen Pflanzen unterschiedliche Temperaturen. Um die das Absterben
von Pflanzen durch falsche Temperaturen zu verhindern, benötigen Roboter zur
Überprüfung Temperatursensoren. Es gibt verschiedene Arten von
Temperatursensoren:

\begin{description}
	\item {Temperatursensoren:}
	      \begin{itemize}
		      \item {\textit{IR}}\\
					Eine der besten Möglichkeiten, Temperaturen zu messen, ist die Infrarottechnik.
Sie bietet einige erhebliche Vorteile, durch die Fähigkeit zur berührungslosen
Temperaturmessung. Zum einen wird der zu messende Gegenstand in keinerlei Weise
beeinflusst, was zum Beispiel die Gefahr der physischem Zerstörung, durch
Berühren empfindlicher Gegenstände, wie Blätter verhindert. Die
Entfernung zum Messpunkt ermöglicht auch, sehr hohe Temperaturen zu messen,
ohne die Infrarotsensorik durch zu hohe Temperaturen zu gefährden.

\begin{figure}[ht]
	\centering
	\includegraphics[width=0.5\textwidth]{bilder/infrarotsensor.png}
	\caption[Sprühdrohne]{Sprühdrohne beim Aufbringen von Pestiziden}
	\label{fig:sprühdrohne}
\end{figure}

\begin{figure}[!h]
	\centering
	\includegraphics[width=0.5\textwidth]{bilder/infrarot_pflanze.jpg}
	\caption[Sprühdrohne]{Sprühdrohne beim Aufbringen von Pestiziden}
	\label{fig:sprühdrohne}
\end{figure}

Funktionsweise von Infrarotthermographie: \\
Messgegenstände mit einer Temperatur von $>$ 900K strahlen Energie ab, der in mithilfe von Wärmebildtechnik sichtbar gemacht werden kann und so als Bild, für den Menschen sichtbar dargestellt werden kann.
Hierzu werden die Spektralbereiche Nahinfrarot(NIR), mittleres Infrarot (MIR) und langes Infrarot (LIR) betrachtet. 
Diese Spektren teilen sich in drei Teile im Bereich von 780 bis 1400nm auf.\cite{schuster2004infrarotthermographie}

		      \item {\textit{Thermistoren}}\\
		            Thermistoren sind Bauteile, welche ihren Widerstand bei steigender Temperatur
verringern. Sie bestehen aus Keramik oder Polymeren und die Temperatur wird
hierbei genau und schnell ausgegeben. Der Standardwiderstand bezieht sich
bei Thermistoren immer auf 25°C. Man Unterscheidet zwischen NTC und PTC, wobei
sich die Buchstaben hier auf das Verhalten des Widerstands bei Wärmeänderung
beziehen. Bei einem NTC sinkt der Widerstand bei steigender Temperatur, bei
einem PTC wird er höher. Meist werden NTC für Temperaturmessungen verwendet.\cite{hering2018temperaturmesstechnik}
		      \item {\textit{Thermoelemente}}\\
		            Ein weiterer Sensor, welcher für die Bodentemperaturmessung verwendet wird,
sind Thermoelemente. Grundlage dieses Sensors ist eine Verbindung zweier Drähte
unterschiedlicher Materialien. 


Die Funktionsweise des Sensors basiert auf
thermoelektrischen Effekten. Der sogenannte Peltier-Effekt besagt hierbei, dass
bei einer Verbindung von zwei, unter Strom stehenden Drähten aus
unterschiedlichen Materialien ein Wärmestrom fließt. Dieser wird an der
Verbindungsstelle absorbiert, was dort zu einer Temperaturveränderung des
Materials führt. Der Seebeck-Effekt geht auf den Stromfluss bei einer
Temperaturänderung ein. Bei der Verbindung der zwei Leiter kommt es zu einem
Stromfluss, wenn an beiden Verbindungsstellen unterschiedliche Temperaturen
anliegen. \cite{bernhard2014thermoelemente}



	      \end{itemize}
\end{description}
		\section{Bodenfeuchtigkeitssensor}
				Pflanzen brauchen Wasser. Je nach Sorte mehr oder weniger, nicht zu viel und nicht zu wenig. Um zu wissen, ob bewässert werden soll, benötigt der Roboter Bodenfeuchtigkeitssensoren.
		
		\begin{description}
			\item {Temperatursensoren:}
			\begin{itemize}
				\item {\textit{kapazitive Sensoren}}
				ÄNDERUNG DER DIELEKTRIZITÄTSKONSTANTE
				
				\cite{induuxwiki2022}
				\item {\textit{Leitfähigkeitssensoren}}
				Wie die meisten bereits wissen, leitet Wasser Strom. Natürlich lässt sich somit über die Leitfähigkeit der Gehalt an Wasser im Boden feststellen.\\
				Das heißt, man leitet sich wie bei Thermistoren und RTDs die Größe über den Widerstand her. Das ist auch der Grund, weshalb es viele Sensoren gibt, die Feuchtigkeit und Temperatur gleichzeitig ermitteln können.
			\end{itemize}
		\end{description}
		
		\section{Nahinfrarotspektroskopie}
		Nahinfrarotspektroskopie(NIR) verwendet Wellenlängen zwischen 780 und 2500nm,
was sich zwischen dem mittleren Infrarotbereich und dem sichtbaren
Spektralbereich befindet. Die Besonderheit bei NIR ist, dass dort die
Oberton-und Kombinationsschwingungen von Molekülen betrachtet werden.
\cite{shenk2001application} Durch die Reaktion der Moleküle auf NIR-Strahlung,
kann man viele Rückschlüsse auf die chemische Zusammensetzung von Materialien
ziehen. Bei einer Wellenlänge von 880nm kann man zum Beispiel im dritten
Oberton eine Reaktion von Fett in der Milch erkennen. \cite{cen2007theory}
Diese Technologie wird in der Landwirtschaft zum Beispiel zum Überwachen von
Futtermischverhältnissen verwendet, da man mithilfe von NIR, wie bereits
erwähnt, den Anteil an Fett, sowie den Anteil an Proteinen bestimmen kann, ohne
das Futter zu beschädigen. Ein weiteres Anwendungsgebiet ist die Getreideernte,
wobei hier der NIR-Sensor meist am Korntankrohr des Mähdreschers verbaut ist.
(Abbildung \ref{fig:Mähhdrescher NIR-Sensor}) 

\begin{figure}[ht]
	\centering
	\includegraphics[width=0.7\textwidth]{bilder/Krone NIR Control dual.jpg}
	\caption[Mähhdrescher mit NIR-Sensor]{NIR-Sensor am Mähdrescher\cite{Krone}}
	\label{fig:Mähhdrescher NIR-Sensor}
\end{figure}

Er dient in diesem Fall der Überwachung der Feuchtigkeit des Saatguts,
um dieses vor Schimmel zu schützen.
	
	%%%%%%%%%%% ANWENDUNGEN %%%%%%%%%%%%%%%%%
	\chapter{Anwendungen in der Landwirtschaft}
		\section{Saat und Ernte}
		Den meisten Teil des Tages verbringt der Landwirt immer noch auf dem Feld. Die
einfachste Art, hinsichtlich der Feldarbeit wäre eine Automatisierung der
Fahrzeuge. Das heißt ein mobiler Roboter mit Ketten- oder Reifenantrieb, würde
das Feld abfahren und die Saat ausbringen und später im Jahr wieder ernten. \\
Hierfür muss der Roboter einige elementare Dinge können. Er muss auch bei
schwierigen Bodenbedinungen gerade Linien auf dem Feld fahren können. Hierfür
kann die Antriebsart angepasst, sowie die Reifen durch Ketten ersetzt werden. Eine
weitere Herausforderung besteht in der Ausbringung des Saatguts in den Boden.
Die einfachste Art wäre das Bohren eines Lochs, in welches das Saatkorn fällt.
Ein trockener, steiniger Boden hat jedoch eine andere Beschaffenheit als
ein feuchter erdiger Boden. Dafür müssen unterschiedliche Aktoren zur Verfügung
stehen. Mit einem Kraftsensor kann der Roboter die Kraft messen, welche er
benötigt, um in den Boden vorzudringen. Anhand dessen kann er ein Bohrgerät für
die Löcher wählen. Diese verschiedenen Bohrer können zum Beispiel direkt am
Roboter angebracht sein.\cite{naik2016precision} Ein wichtiger für das Sähen benötigter Sensor ist ein
Ultraschallsensor, zum Messen der Bodenentfernung. Damit können die Samen auf
die exakte Tiefe in den Boden eingebracht werden.\\ Beim Ernten kommt es
vorallem auf die Sorte an. Hierbei kommt es weniger auf die Sensorik, als auf
die Aktorik an. Das liegt daran, dass alle Pflanzen bei der Ernte ziemlich
gleich groß sind, innerhalb ihrer Sorte, welche im Vornherein bekannt ist.
Dabei werden die Pflanzen komplett "abgeschnitten". Man könnte jedoch bereits
beim Erntevorgang durch eine Bildverarbeitung beschädigte, kaputte Ernte
aussortieren. Hierfür wäre vor allem ein gutes Kamerasystem nötig, wobei jedoch der Kosten-Nutzen-Faktor hierbei beachtet werden sollte, da ein komplexes Kamerasystem
schnell hohe Kosten verursacht.
		\section{Tierpflege}
		Der Großteil, neben der Feldarbeit, dreht sich bei der Landwirtschaft um den
Viehbetrieb. Hier gibt es bereits einige Automatisierungen, wie zum Beispiel
Melkstraßen, wo Kühe mithilfe von Melkrobotern nacheinander vollautomatisch
gemolken werden. Was zudem sehr leicht zu automatisieren ist, ist die Ernährung
jeglicher Tierarten.\\ Man benötigt lediglich ein System, das Futter von großen
Lagerbehältern zu den Ställen/Gehegen bringt. Das kann je nach Bauernhof
unterschiedlich aussehen. Die einfachste Umsetzung wäre hierbei meiner Hinsicht
nach ein höher gelegenes Schienensystem. Damit gibt es keine Behinderung der
sonstigen Arbeit, und man kann das komplette System mittels eines
Mikrokontrollers, wie zum Beispiel eines Raspberrys, steuern. Die
herkömmlichsten Roboter gleichen jedoch herkömmlichen Futtermischwägen, welche
vollautomatisiert durch die Ställe fahren. (Abb. 5.1)

\begin{figure}[ht]
    \centering
    \includegraphics[width=0.7\textwidth]{bilder/futterroboter.jpg}
    \caption[Futterroboter in Kuhstall]{Futterroboter in Kuhstall}
    \label{fig:futterroboter}
\end{figure}	
		\section{Bewässerung}
		Eine weitere Anwendung, welche in der Landwirtschaft in Kombination mit
Sensorik viel Einsatz findet ist Bewässerung. Vorallem durch das immer weniger
werdende Grundwasser, sind Pflanzen auf eine externe Bewässerung angewiesen.
\cite{liu2018self} Bewässerung kann auf unterschiedliche Arten realisiert
werden. Man unterscheidet zwischen Flächenbewässerung und punktueller
Bewässerung. Flächenbewässerung wird meist mit Sprenklern realisiert die mittig
aufgestellt werden und so die Pflanzten von oben mit Wasser versorgen. Vorallem
in sehr heißen Gebieten ist das recht ineffizient, da ein Großteil des Wassers
auf der Pflanze verdunstet, bevor es aufgenommen werden kann. Ein sehr viel
besseres System für die Freiflächenanwendung sind Schläuche, welche in
Bodennähe verlegt werden. Mit vereinzelten Löchern wird durch den Schlauch
Wasser abgegeben, welches direkt vom Boden aufgenommen werden kann und unter
dem Pflanzen, geschützt vor der Sonne, nicht verdunstet. Der Nachteil an
Schläuchen im Freien ist die Anfälligkeit gegenüber Zerstörung durch Tiere,
besonders Marder neigen dazu, Schläuche aufzubeißen, womit eine Verluststelle
im System besteht, die nur schwer bemerkt und lokalisiert werden kann. Einen
weiteren Aspekt, den es zu beachten gilt, wenn Felder in freier Umgebung
bewässert werden sollen, ist die dauerhafte Überwachung der Bodenfeuchtigkeit.
Wenn es regnet, neigt ein bewässertes System schnell zur Überwässerung. Durch
Bodenfeuchtigkeitssensoren (Kapitel 4.2) kann mithilfe eines Mikrokontrollers
ein Feuchtigkeitsthreshold eingestellt werden, welcher sicherstellt, dass die
Pflanzen genau die richtige Menge an Wasser bekommen. Außerdem gibt es Ansätze,
zur passiven Versorgung mit Wasser, wobei man versucht, das Optimum aus
Niederschlag und Grundwasser zu nutzen. Ein Beispiel hierfür beschäftigt sich
mit der Idee ein T-Stück im Boden einzusetzen. Der obere Teil des T-Stücks soll
das schnelle versickern des Niederschlags verhindern, womit die Pflanzen mehr
Zeit haben diesen mithilfe der Wurzeln zu verwenden. Der untere Teil des Teils
reicht bis zum Grundwasser. Wie bereits vorher erwähnt, ist der
Grundwasserspiegel inzwischen zu niedrig, um von den Pflanzen erreicht zu
werden. Mithilfe von kleinen Röhren wird die Kapillarkraft verwendet, um
Grundwasser von unten, nach oben zu den Pflanzen zu transportieren, wobei das
vollkommen autonom und ohne jegliche Ansteuerung funktioniert. \cite{liu2018self}	
		\section{Schädlingsbekämpfung}
		Meist werden Schädlinge, wie Parasiten und Unkraut mithilfe von Pestiziden
bekämpft. Diese werden meist unter Zuhilfenahme von Traktoren auf dem Feld
ausgebracht. Ein neuerer Ansatz ist die Verteilung mit Drohnen, die wie bereits
erwähnt die Flüssigkeit aus der Höhe weitflächig auf den Feldern verteilen
könnten. Jedoch ist aufgrund der Gefahr von Verwehung die maximale Flughöhe auf
2m über den Pflanzen, sowie die Geschwindigkeit auf 13km/h
beschränkt.\cite{bvl} Somit erübrigt sich dieser Vorteil, was jedoch bleibt ist
die fehlende Bodenzerstörung durch Reifen, welche bei einer traktorbasierten
Ausbringung der Fall wäre. Auch in der Atemluft lagern sich Pestizide ab, was
bereits 2019 in einer Studie nachfgewiesen wurde, in welcher Baumrinde
untersucht wurde, wobei man in 47 Proben 104 verschiedene Pestizide finden
konnte.\cite{clausing2020baumrinden}

	%%%%%%%%%%%% HERAUSFORDERUNGEN %%%%%%%%%%
	\chapter{Künstliche Intelligenz}
	Wie auch in allen Bereichen, ist auch in der Landwirtschaft gerade eine sehr
starke Entwicklung von künstlicher Intelligenz zu erkennen. Künstliche
Intelligenz wird dabei verwendet, um große Datenmengen von Sensoren in der
Landwirtschaft zu nutzen, um neuronale Netzte zu trainieren. Meist findet KI
ihren Einsatz im Bereich der Schädlingsbekämpfung oder im Ausbringen von Saat.
\cite{mci/Mohr2020} Algorithmen zur künstlichen Intelligenz werden zum Beispiel
verwendet, um den Lebenszyklus einer Pflanze zu überwachen. Kontrolliert werden
kann dieser Prozess über viele verschiedene Arten, zum Beispiel Drohnen,
Satelliten oder Ultraleichtflugzeugen. Ein gutes Beispiel hierfür ist das Start-up Taranis. \cite{wennker2020kunstliche}
	%%%%%%%%%%% FAZIT %%%%%%%%%%%%%%%%%%%%%%%
	\chapter{Fazit}
	Zusammenfassend kann man davon sprechen, dass auch die Landwirtschaft von dem
Umschwung in die Automatisierung von Prozessen betroffen ist. Vorallem in
dieser Branche gibt es viele gute Möglichkeiten, Systeme mithilfe von Sensoren
zu überwachen und zu beeinflussen. Auch mit dem Hintergrund des
Bevölkerungswachstums und der damit einhergehenden Steigerung des Mangel an
Lebensmitteln ist es besonders wichtig, jede Chance zu nutzen, die
Landwirtschaft so effizient wie möglich zu gestalten. Systeme wie der Farmbot
helfen, jeden freien Platz sinnvoll zu nutzen und animieren die Menschen, auch
Zuhause Pflanzen selbst zu züchten. Jedoch muss dies in einem für die Umwelt
möglichst erträglichen Rahmen geschehen. Hierfür sind die Roboter eine große
Hilfe, die zum Großteil, im Gegensatz zu den alten dieselbetriebenen
Landmaschinen, durch elektrische Antriebe bewegt werden. Durch Solarmodule
können diese immer weiter nahezu autark und emissionsfrei arbeiten, was eine
hohe Entlastung für die Umwelt darstellt.
	%%%%%%%%%%% INHALTSVERZEICHNIS %%%%%%%%%%
	\printbibliography
	%%%%%%%%%%% ABBILDUNGSVERZEICHNIS %%%%%%%
	\listoffigures
\end{document}
%% im fließtext bei Zitat auf Autor beziehen %%