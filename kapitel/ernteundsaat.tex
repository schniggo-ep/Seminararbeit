Den meisten Teil des Tages verbringt der Landwirt immernoch auf dem Feld. Die
einfachste Art, hinsichtlich der Feldarbeit wäre eine Automatisierung der
Fahrzeuge. Das heißt ein mobiler Roboter mit Ketten- oder Reifenantrieb, würde
das Feld abfahren und die Saat ausbringen und später im Jahr wieder ernten. \\
Hierfür muss der Roboter einige elementare Dinge können. Er muss auch bei
schwierigen Bodenbedinungen gerade Linien auf dem Feld fahren können. Hierfür
kann man die Antriebsart anpassen, sowie die Reifen durch Ketten ersetzen. Eine
weitere Herausforderung besteht in der Ausbringung des Saatguts in den Boden.
Die einfachste Art wäre das Bohren eines Lochs, in welches das Saatkorn fällt.
Ein sehr trockener, steiniger Boden hat jedoch eine andere Beschaffenheit als
ein feuchter erdiger Boden. Dafür müssen unterschiedliche Aktoren zur Verfügung
stehen. Mit einem Kraftsensor kann der Roboter die Kraft messen, welche er
benötigt, um in den Boden vorzudringen. Anhand dessen kann er ein Bohrgerät für
die Löcher wählen. Diese verschiedenen Bohrer können zum Beispiel direkt am
Roboter angebracht sein.\cite{naik2016precision} Ein wichtiger für das Sähen benötigter Sensor ist ein
Ultraschallsensor, zum Messen der Bodenentfernung. Damit können die Samen auf
die exakte Tiefe in den Boden eingebracht werden.\\ Beim Ernten kommt es
vorallem auf die Sorte an. Hierbei kommt es weniger auf die Sensorik, als auf
die Aktorik an. Das liegt daran, dass alle Pflanzen bei der Ernte ziemlich
gleich groß sind, innerhalb ihrer Sorte, welche im Vornherein bekannt ist.
Dabei werden die Pflanzen komplett "abgeschnitten". Man könnte jedoch bereits
beim Erntevorgang durch eine Bildverarbeitung beschädigte, kaputte Ernte
aussortieren. Hierfür wäre vorallem ein sehr gutes Kamerasystem nötig, wobei
man jedoch Kosten-Nutzen hierbei beachten sollte, da ein komplexes Kamerasystem
schnell sehr teuer werden kann.