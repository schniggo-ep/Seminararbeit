Den meisten Teil des Tages verbringt der Bauer immer noch auf dem Feld. Die einfachste Art hierfür wäre eine Automatisierung der Fahrzeuge. Das heißt ein mobiler Roboter mit Ketten- oder Reifenantrieb, würde durch fest übergebene GPS-Daten Bahnen im Feld abfahren und ernten, bzw. sähen. \\
Ein wichtiger für das Sähen benötigter Sensor ist ein Ultraschallsensor, zum Messen der Bodenentfernung. Damit können die Samen auf die exakte Tiefe in den Boden eingebracht werden.\\
Beim Ernten kommt es vorallem auf die Sorte an. Hierbei kommt es weniger auf die Sensorik, als auf die Aktorik an. Das liegt daran, dass alle Pflanzen bei der Ernte ziemlich gleich groß sind, innerhalb ihrer Sorte, welche im Vornherein bekannt ist. Dabei werden die Pflanzen komplett "abgeschnitten". Man könnte jedoch bereits beim Erntevorgang durch eine Bildverarbeitung beschädigte, kaputte Ernte aussortieren. Hierfür wäre vorallem ein sehr gutes Kamerasystem nötig, wobei man jedoch Kosten-Nutzen hierbei beachten sollte, da ein komplexes Kamerasystem schnell sehr teuer werden kann.