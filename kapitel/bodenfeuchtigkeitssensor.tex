Je nach Sorte brauchen Pflanzen mehr oder weniger Wasser, nicht zu viel und
nicht zu wenig. Um zu wissen, ob bewässert werden soll, benötigt der Roboter
Bodenfeuchtigkeitssensoren.

\begin{description}
	\item {Feuchtigkeitssensoren:}
	      \begin{itemize}
		      \item {\textit{kapazitive Sensoren}}\\
		            Kapazitive Feuchtigkeitssensoren bestehen aus zwei Elektroden und einem
Dielektrikum, in welches bei porösen Elektroden Wassermoleküle diffundieren.
Dieser Prozess ist abhängig von der relativen Feuchte. Da die Permitivität von
Wasser hoch ist steigt die Kapazität des Sensors, was, mithilfe eines
Oszillators in eine Frequenz umgewandelt, gemessen werden kann. Kapazitive
Sensoren können gut durch bestimmte salzhaltige Lösungen kontrolliert, und bei
Fehlern korrigiert werden und sollte auch bei längerem Einsatz gemacht
werden.\cite{koch2006neues}
		      \item {\textit{Leitfähigkeitssensoren}}\\
		            Diese Sensoren bestehen in der Regel aus einem feuchtigkeitsempfindlichen Material, das zwischen zwei Elektroden platziert ist. 
					Wenn Feuchtigkeit in das Material eindringt, verändert sich die elektrische Leitfähigkeit und somit auch der Widerstand. 
					Je feuchter das Material ist, desto niedriger wird der Widerstand.
		            Um den Feuchtigkeitsgehalt zu messen, wird eine elektrische Spannung an die
		            Elektroden angelegt, und der resultierende Strom oder die Spannung über dem
		            Sensor wird gemessen. Durch die Messung des Widerstands kann auf die
		            Feuchtigkeit geschlossen werden. Dieses Signal kann dann in elektronische
		            Schaltungen oder Mikrocontroller eingespeist und weiterverarbeitet werden.
	      \end{itemize}
\end{description}