		Je nach Sorte brauchen Pflanzen mehr oder weniger Wasser, nicht zu viel und nicht zu wenig. Um zu wissen, ob bewässert werden soll, benötigt der Roboter Bodenfeuchtigkeitssensoren.
		
		\begin{description}
			\item {Feuchtigkeitssensoren:}
			\begin{itemize}
				\item {\textit{kapazitive Sensoren}}\\
				Kapazitive Feuchtigkeitssensoren bestehen aus zwei Elektroden und einem
Dielektrikum, in welches bei porösen Elektroden Wassermoleküle diffundieren.
Dieser Prozess ist abhängig von der relativen Feuchte. Da die Permitivität von
Wasser hoch ist steigt die Kapazität des Sensors, was, mithilfe eines
Oszillators in eine Frequenz umgewandelt, gemessen werden kann. Kapazitive
Sensoren können gut durch bestimmte salzhaltige Lösungen kontrolliert, und bei
Fehlern korrigiert werden und sollte auch bei längerem Einsatz gemacht
werden.\cite{koch2006neues}
				\item {\textit{Leitfähigkeitssensoren}}\\
				Wie die Meisten bereits wissen, leitet Wasser Strom. Natürlich lässt sich somit über die Leitfähigkeit der Gehalt an Wasser im Boden feststellen.\\
				Das heißt, man leitet sich wie bei Thermistoren und RTDs die Größe über den Widerstand her. Das ist auch der Grund, weshalb es viele Sensoren gibt, die Feuchtigkeit und Temperatur gleichzeitig ermitteln können.
			\end{itemize}
		\end{description}
		