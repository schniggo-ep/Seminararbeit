In den vergangenen Jahrzehnten entwickelte sich durch das steigende
Bevölkerungswachstum ein weltweiter Mangel an Ressourcen, vor allem an
Nahrungsmitteln.\\Die Folge
hiervon sind Massentierhaltung, Monokulturen und Gewächshausplantagen. Der
Trend geht zur Perfektion der Erträge, bei möglichst geringem Aufwand. Dabei
wird der Fokus darauf gelegt, die vorhandene Fläche optimal zu nutzen, um viele
Lebensmittel auf geringem Raum zu produzieren. Dabei wird außerdem darauf
geachtet, den Ertrag der vorhandenen Pflanzen zu maximieren. Dies kann durch
eine intelligente Bewässerung, die Optimierung der Temperaturbedingungen, sowie
eine effiziente Bekämpfung von Unkraut und Schädlingen realisiert werden. Dabei
spielen Sensoren, zur Überwachung der Felder eine große Rolle. Aus den
Sensordaten können Robotersysteme erkennen, an welcher Stelle eine Aktion zur
Verbesserung Pflanzenbedingungen und damit der Maximierung des Ertrags
erforderlich ist. Ähnlich verhält es sich auch bei der Tierzucht, wobei Roboter
hierbei auch zum Wohl der Tiere durch Futterverpflegung und Stallreinigung
beigetragen wird. \\Auch in anderen Bereichen ist die Frage nach schnellen,
zuverlässigen Massenproduktionen bereits seit mehreren Jahrzehnten diskutiert.
\\Vor allem die Automobilindustrie hat bereits früh begonnen sich mit
Automatisierung zu beschäftigen. Erste Ansätze gab es hier bereits 1913\cite{raff1987did} durch
Henry Ford, welcher die ersten Autos auf dem Fließband zusammenbauen ließ.
Dieses System wurde über die Jahrzehnte soweit ausgebaut, dass heutzutage
bereits viele Industrieroboter die Bandarbeit nahezu komplett autonom
erledigen. \\Jedoch gibt es auch in der Landwirtschaft bereits Ansätze, um die
Arbeit zu industrialisieren. \\In der folgenden Arbeit werden die fundamentalen
Grundlagen für eine effiziente Nutzung von sensorbasierten Robotersystemen
erläutert. Dabei wird auf bereits existierende Systeme aufgezeigt, die
benötigte Sensorik erläutert, sowie auf gesellschaftliche und technische
Herausforderungen eingegangen.