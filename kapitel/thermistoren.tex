Thermistoren sind Bauteile, welche ihren Widerstand bei steigender Temperatur
verringern. Sie bestehen aus Keramik oder Polymeren und die Temperatur wird
hierbei genau und schnell ausgegeben. Der Standardwiderstand bezieht sich
bei Thermistoren immer auf 25°C. Man Unterscheidet zwischen NTC und PTC, wobei
sich die Buchstaben hier auf das Verhalten des Widerstands bei Wärmeänderung
beziehen. Bei einem NTC sinkt der Widerstand bei steigender Temperatur, bei
einem PTC wird er höher. Meist werden NTC für Temperaturmessungen verwendet.\cite{hering2018temperaturmesstechnik}