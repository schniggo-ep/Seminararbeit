Wie auch in allen Bereichen, ist auch in der Landwirtschaft gerade eine sehr
starke Entwicklung von künstlicher Intelligenz zu erkennen. Künstliche
Intelligenz wird dabei verwendet, um große Datenmengen von Sensoren in der
Landwirtschaft zu nutzen, um neuronale Netzte zu trainieren. Meist findet KI
ihren Einsatz im Bereich der Schädlingsbekämpfung oder im Ausbringen von Saat.
\cite{mci/Mohr2020} Algorithmen zur künstlichen Intelligenz werden zum Beispiel
verwendet, um den Lebenszyklus einer Pflanze zu überwachen. Kontrolliert werden
kann dieser Prozess über viele verschiedene Arten, zum Beispiel Drohnen,
Satelliten oder Ultraleichtflugzeugen. Ein gutes Beispiel hierfür ist das Start-up Taranis. \cite{wennker2020kunstliche}