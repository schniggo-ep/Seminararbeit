Kapazitive Feuchtigkeitssensoren bestehen aus zwei Elektroden und einem
Dielektrikum, in welches bei porösen Elektroden Wassermoleküle diffundieren.
Dieser Prozess ist abhängig von der relativen Feuchte. Da die Permitivität von
Wasser hoch ist steigt die Kapazität des Sensors, was, mithilfe eines
Oszillators in eine Frequenz umgewandelt, gemessen werden kann. Kapazitive
Sensoren können gut durch bestimmte salzhaltige Lösungen kontrolliert, und bei
Fehlern korrigiert werden und sollte auch bei längerem Einsatz gemacht
werden.\cite{koch2006neues}