Wie auch in allen Bereichen, ist auch in der Landwirtschaft gerade eine sehr
starke Entwicklung von künstlicher Intelligenz zu erkennen. Künstliche
Intelligenz wird dabei verwendet, um große Datenmengen von Sensoren in der
Landwirtschaft zu nutzen, um neuronale Netze zu trainieren. Meist findet KI
ihren Einsatz im Bereich der Schädlingsbekämpfung oder im Ausbringen von
Saat.\cite{mci/Mohr2020} Algorithmen zur künstlichen Intelligenz werden zum
Beispiel verwendet, um den Lebenszyklus einer Pflanze zu überwachen.
Kontrolliert werden kann dieser Prozess über viele verschiedene Arten, zum
Beispiel Drohnen, Satelliten oder Ultraleichtflugzeugen. Ein gutes Beispiel
hierfür ist das Start-up Taranis.Taranis benutzt Drohnen, welche mit bis zu
200 km/h extrem hochauflösende Aufnahmen von Feldern machen können. In
Verbindung mithilfe von Satelliten und Flugzeugen überwachen sie so 80.000
Quadratkilometer Agrarfläche in mehreren Ländern. Diese Pflanzen werden
überwacht, und der Landwirt wird sofort informiert, sobald eine Pflanze
beschädigt wird. Somit kann dieser die Pflanze im Frühstadium sofort entfernen
und noch zur richtigen Zeit eine Neue anpflanzen. Zudem können die
Drohnenbilder mithilfe der von Tensorflow unterstützen KI durch ihre hohe
Auflösung sogar Insekten und andere Schädling erkennen, bevor die Pflanzen von
diesen zerstört werden. Diese Technologie wird in Zukunft noch an Bedeutung
zunehmen, da bereits jetzt 20-40 Prozent der Pflanzen durch Unkraut und
ähnliches absterben. Zudem wird die Wettervorhersage immer mehr durch den
Einsatz von künstlicher Intelligenz verbessert. Durch die Vielzahl an
Satelliten steht eine extreme Menge an Wetterdaten zur Verfügung, die
regelmäßig geupdated wird. Da die Ernte sehr anfällig gegenüber Gewitter,
speziell Hagel ist, haben mehrere Projekte, wie zum Beispiel das US National
Oceanic and Atmospheric Administration bereits begonnen, sich mit dem regionen-
und zeitabhängigen Wetterprognosen zu befassen.\cite{wennker2020kunstliche}