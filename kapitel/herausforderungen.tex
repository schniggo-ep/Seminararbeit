Bei allen Vorteilen, welche Robotik in der Landwirtschaft bietet gibt es auch Nachteile. 
Aufgaben werden von autonomen Systemen übernommen, womit auch die Arbeiter ihre Aufgabe an die Roboter verlieren, was bedeutet dass viele Erntehelfer, Hofmitarbeiter umschulen müssen, oder sich komplett einen anderen Beruf suchen müssen.
Zudem sind die Farmbetriebe damit zunehmend abhängig von Technik. Das erfordert immer größere Kenntnisse bei den Bauern über komplexe technische Systeme, um technische Defekte im Zweifel selbst beheben zu können.
Andernfalls können Systemausfälle in dieser Branche vorallem für die einzelnen Betriebe extreme Folgen haben, wenn man zum Beispiel an das Verkommen einer kompletten Jahresernte aufgrund von kaputter Erntesysteme denkt.