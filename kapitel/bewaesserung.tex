Eine weitere Anwendung, welche in der Landwirtschaft in Kombination mit
Sensorik viel Einsatz findet ist Bewässerung. Vorallem durch das immer weniger
werdende Grundwasser, sind Pflanzen auf eine externe Bewässerung angewiesen.
\cite{liu2018self} Bewässerung kann auf unterschiedliche Arten realisiert
werden. Man unterscheidet zwischen Flächenbewässerung und punktueller
Bewässerung. Flächenbewässerung wird meist mit Sprenklern realisiert die mittig
aufgestellt werden und so die Pflanzten von oben mit Wasser versorgen. Vorallem
in sehr heißen Gebieten ist das recht ineffizient, da ein Großteil des Wassers
auf der Pflanze verdunstet, bevor es aufgenommen werden kann. Ein sehr viel
besseres System für die Freiflächenanwendung sind Schläuche, welche in
Bodennähe verlegt werden. Mit vereinzelten Löchern wird durch den Schlauch
Wasser abgegeben, welches direkt vom Boden aufgenommen werden kann und unter
dem Pflanzen, geschützt vor der Sonne, nicht verdunstet. Der Nachteil an
Schläuchen im Freien ist die Anfälligkeit gegenüber Zerstörung durch Tiere,
besonders Marder neigen dazu, Schläuche aufzubeißen, womit eine Verluststelle
im System besteht, die nur schwer bemerkt und lokalisiert werden kann. Einen
weiteren Aspekt, den es zu beachten gilt, wenn Felder in freier Umgebung
bewässert werden sollen, ist die dauerhafte Überwachung der Bodenfeuchtigkeit.
Wenn es regnet, neigt ein bewässertes System schnell zur Überwässerung. Durch
Bodenfeuchtigkeitssensoren (Kapitel 4.2) kann mithilfe eines Mikrokontrollers
ein Feuchtigkeitsthreshold eingestellt werden, welcher sicherstellt, dass die
Pflanzen genau die richtige Menge an Wasser bekommen. Außerdem gibt es Ansätze,
zur passiven Versorgung mit Wasser, wobei man versucht, das Optimum aus
Niederschlag und Grundwasser zu nutzen. Ein Beispiel hierfür beschäftigt sich
mit der Idee ein T-Stück im Boden einzusetzen. Der obere Teil des T-Stücks soll
das schnelle versickern des Niederschlags verhindern, womit die Pflanzen mehr
Zeit haben diesen mithilfe der Wurzeln zu verwenden. Der untere Teil des Teils
reicht bis zum Grundwasser. Wie bereits vorher erwähnt, ist der
Grundwasserspiegel inzwischen zu niedrig, um von den Pflanzen erreicht zu
werden. Mithilfe von kleinen Röhren wird die Kapillarkraft verwendet, um
Grundwasser von unten, nach oben zu den Pflanzen zu transportieren, wobei das
vollkommen autonom und ohne jegliche Ansteuerung funktioniert. \cite{liu2018self}