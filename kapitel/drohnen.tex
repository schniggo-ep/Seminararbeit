In der Landwirtschaft geht es häufig darum, Sachen (Flüssigkeiten,Samen,etc.) auf dem Feld auszubringen. Auf die herkömmliche Art macht das ein Bauer, mithilfe eines Traktors. Je nach Aufgabe und Feldgröße benötigt ein Landwirt hierfür mehrere kostbare Stunden. Die Ausbringung von Flüssigkeiten wäre ein typischer Anwendungsfall für eine Drohne.

\begin{figure}[ht]
	\centering
	\includegraphics[width=0.5\textwidth]{bilder/drohne_im_sprüheinsatz.png}
	\caption[Sprühdrohne]{Sprühdrohne}
	\label{fig:sprühdrohne}
\end{figure}

Die Drohne kann hierbei aus geringer Höhe die entsprechenden Flüssigkeiten über eine Düse ausbringen. Das kann durch Routenplanung komplett automatisiert passieren. Lediglich ein Flüssigkeitstank ist hier benötigt, welcher jedoch innerhalb weniger Minuten vom Landwirt an einer fest vorgegebenen Stelle platziert werden kann. Somit hat die Drohne einen festen Bezugspunkt für das Auffüllen der Flüssigkeit, ähnlich einer Ladestation eines Rasenmähers.
