Meist werden Schädlinge, wie Parasiten und Unkraut mithilfe von Pestiziden
bekämpft. Diese werden meist unter Zuhilfenahme von Traktoren auf dem Feld
ausgebracht. Ein neuerer Ansatz ist die Verteilung mit Drohnen, die wie bereits
erwähnt die Flüssigkeit aus der Höhe weitflächig auf den Feldern verteilen
könnten. Jedoch ist aufgrund der Gefahr von Verwehung die maximale Flughöhe auf
2m über den Pflanzen, sowie die Geschwindigkeit auf 13km/h
beschränkt.\cite{bvl} Somit erübrigt sich dieser Vorteil, was jedoch bleibt ist
die fehlende Bodenzerstörung durch Reifen, welche bei einer traktorbasierten
Ausbringung der Fall wäre. Auch in der Atemluft lagern sich Pestizide ab, was
bereits 2019 in einer Studie nachgewiesen wurde, in welcher Baumrinde
untersucht wurde, wobei das Institut "TIEM Integrierte Umweltüberwachung GbR in
47 Proben 104 verschiedene Pestizide finden
konnte.\cite{clausing2020baumrinden} Für fliegende Schädlinge wurde bereits ein
Konzept für Gewächshäuser entwickelt. Dieses besteht aus einem Roboterarm,
welcher mit Klebeflächen ausgestattet ist. Dieser bewegt sich über die
Pflanzen, während die Insekten mit Druckluft aufgeschreckt werden und daraufhin
in die Klebeflächen fliegen und dort festkleben. Der Aufbau ist in Abbildung
\ref{fig:schaedlingsbekaempfung} nochmals exemplarisch dargestellt:\cite{hoing2020entwicklung}

\begin{figure}[!h]
    \centering
    \includesvg[width=0.7\textwidth]{bilder/schaedlingsbekaempfung.svg}
    \caption{Roboter zur Schädlingsbekämpfung}
    \label{fig:schaedlingsbekaempfung}
\end{figure}

