Zusammenfassend kann man davon sprechen, dass auch die Landwirtschaft von dem
Umschwung in die Automatisierung von Prozessen betroffen ist. Vorallem in
dieser Branche gibt es viele gute Möglichkeiten, Systeme mithilfe von Sensoren
zu überwachen und zu beeinflussen. Auch mit dem Hintergrund des
Bevölkerungswachstums und der damit einhergehenden Steigerung des Mangel an
Lebensmitteln ist es besonders wichtig, jede Chance zu nutzen, die
Landwirtschaft so effizient wie möglich zu gestalten. Systeme wie der Farmbot
helfen, jeden freien Platz sinnvoll zu nutzen und animieren die Menschen, auch
Zuhause Pflanzen selbst zu züchten. Jedoch muss dies in einem für die Umwelt
möglichst erträglichen Rahmen geschehen. Hierfür sind die Roboter eine große
Hilfe, die zum Großteil, im Gegensatz zu den alten dieselbetriebenen
Landmaschinen, durch elektrische Antriebe bewegt werden. Durch Solarmodule
können diese immer weiter nahezu autark und emissionsfrei arbeiten, was eine
hohe Entlastung für die Umwelt darstellt.