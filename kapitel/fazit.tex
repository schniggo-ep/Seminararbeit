Zusammenfassend kann davon gesprochen werden, dass auch die Landwirtschaft von dem
Umschwung in die Automatisierung von Prozessen betroffen ist. Vor allem in
dieser Branche gibt es viele gute Möglichkeiten, Systeme mithilfe von Sensoren
zu überwachen und zu beeinflussen. Auch mit dem Hintergrund des
Bevölkerungswachstums und der damit einhergehenden Steigerung des Mangel an
Lebensmitteln ist es besonders wichtig, jede Chance zu nutzen, die
Landwirtschaft so effizient wie möglich zu gestalten. Auch den Aspekt, der
Zerstörung des fruchtbaren Boden muss beachtet werden, beim Gedanken an die
Ernährung der immer größer werdenden Menschheit.\cite{rainer2003diskurs}
Systeme wie der Farmbot helfen, jeden freien Platz sinnvoll zu nutzen und
animieren die Menschen, auch Zuhause Pflanzen selbst zu züchten. Jedoch muss
dies in einem für die Umwelt möglichst erträglichen Rahmen geschehen. Hierfür
sind die Roboter eine große Hilfe, die zum Großteil, im Gegensatz zu den alten
dieselbetriebenen Landmaschinen, durch elektrische Antriebe bewegt werden.
Durch Solarmodule können diese immer weiter nahezu autark und emissionsfrei
arbeiten, was eine hohe Entlastung für die Umwelt darstellt. Dass die
konservative Landwirtschaft immer mehr Technik einsetzt, zeigt sich bereits an
einigen Umfragen. 
Auch gesundheitlich kann die Digitalisierung der
Landwirtschaft viele Vorteile bieten. Durch den Verzicht auf Pestizide, bzw.
die Ausbringung durch Drohnen aus der Ferne, wird die Gesundheit der Bauern
geschont, da sie die giftigen Pestizide nicht einatmen.
\cite{jungwirth2022arbeitszeitbedarf}