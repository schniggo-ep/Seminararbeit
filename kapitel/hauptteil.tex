
In der Landwirtschaft gibt es viele Prozesse, die immer gleich sind. Solche Aufgaben sind perfekt dafür geeignet, um mithilfe von Robotik automatisiert zu werden. Zudem hat man bei Feldern, Äckern, etc. immer feste Ortspunkte und kann somit sehr gut mit GPS Koordinaten arbeiten.
Im Folgenden wird auf die Grundlagen der benötigten Sensoren, sowie deren Flächenabdeckungsmöglichkeiten eingegangen.
Zudem werde ich einige bereits existierende Lösungsmöglichkeiten vorstellen und einen Ausblick geben, wie diese auf Industriegröße skaliert werden können.
	
