In der Landwirtschaft gibt es viele Prozesse, die immer gleich sind. Solche
Aufgaben sind perfekt dafür geeignet, um mithilfe von sensorbasierter Robotik
automatisiert zu werden. Zudem hat man bei Feldern, Äckern, etc. immer feste
Ortspunkte und kann somit sehr gut mit den globalen GPS Koordinaten arbeiten.
Im Folgenden wird auf die Grundlagen der benötigten Sensoren, sowie deren
Flächenabdeckungsmöglichkeiten eingegangen. Zudem werde ich einige bereits
existierende Robotersysteme vorstellen und einen Ausblick geben, wie diese auf
Industriegröße skaliert werden können und wo sie bereits Anwendung finden. Im
weiteren Verlauf wird dann noch der immer weitverbreitetere Aspekt der
künstlichen Intelligenz aufgegriffen und der Bezug zur Landwirtschaft
hergestellt. Das Ende der Arbeit fässt die gesammelten Erkenntnisse zusammen
und gibt eine Zukunftsprognose zum Thema Automatisierung der Landwirtschaft
durch sensorbasierte Robotik.